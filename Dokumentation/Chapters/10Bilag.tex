\section{Bilag}
	\subsection{Koden}\label{sec:Koden}
		\begin{lstlisting}
#include <Wire.h>
#include <SPI.h>
#include <RTClib.h>
#include <SD.h>
#include <OneWire.h>
#include <DallasTemperature.h>
#include <ph4502c_sensor.h>
#include <Adafruit_VEML6075.h>
#include <Adafruit_MPL3115A2.h>

RTC_DS1307 rtc;

// RTC config
const int rtcSDA = 20;
const int rtcSCL = 21;
// We desided to take a reading every 10 minutes, but if you want to change it this is the place to do it!
const int delayInMinutes = 10;

// SD card config
const int sdCard = 53;                  // This is the pin for the SD card reader (CS)
const char* dataFileName = "data.csv";  // Name of the file you want to save to.
File dataFile;

/* Analog pin for analog temperature sensor (Couldn't get it to work)
const int termistorPin = A0;
*/

// Digital temperature sensor config (DS18B20)
const int tempDigitalPin = 22;        // Data pin connection for sensor
OneWire oneWire(tempDigitalPin);      // Setup a oneWire instance to communicate with any OneWire devices
DallasTemperature tempSensors(&oneWire);  // Pass our oneWire reference to Dallas Temperature sensor

// PH sensor config
const uint8_t PO = A0;
const uint8_t TO = A1;
PH4502C_Sensor ph4502c(PO, TO);

// UV sensor
Adafruit_VEML6075 uv = Adafruit_VEML6075();

// Pressure sensor
Adafruit_MPL3115A2 baro;

// Gas sensor A2 pin
const uint8_t gasSensorPin = A2;

void setup() {
	Serial.begin(9600);
	
	if (!SD.begin(sdCard)) {
		Serial.println("initialization failed!");
		while (1)
		;
	}
	if (!SD.exists(dataFileName)) {
		// Prints the header to the sd card
		header();
	}
	Serial.println("initialization done.");
	
	// Initializing UV sensor
	if (! uv.begin()) {
		Serial.println("Failed to communicate with VEML6075 sensor, check wiring?");
		while (1) { delay(100); }
	}
	// Initializing barometric sensor
	if (!baro.begin()) {
		Serial.println("Could not find sensor. Check wiring.");
		while(1);
	}
	baro.setSeaPressure(1013.26);
	
	// Initializes the rtc module
	if (!rtc.begin()) {
		Serial.println("Couldn't find RTC");
		while (1)
		;
	}
	
	DateTime now = rtc.now();
	if (now.year() < 2000) {
		Serial.println("RTC lost power, let's set the time!");
		// Following line sets the RTC to the date & time this sketch was compiled
		rtc.adjust(DateTime(F(__DATE__), F(__TIME__)));  // This is the line that sets the time to the time the code was compiled, if you want to change it you can do it here!
	}
	
	// Initializes the digital temperature sensor
	tempSensors.begin();
	
	// Initialize the PH sensor
	ph4502c.init();
}

void loop() {
	time();
}

void time() {
	DateTime now = rtc.now();
	
	int startTime = now.minute();
	while ((startTime + delayInMinutes) % 60 != now.minute()) {
		now = rtc.now();  // Update the current time
		Serial.println("Waiting...");
		delay(1000);
	}
	
	// Keeps the time and date in separate strings
	String date = String(now.year()) + "/" + now.month() + "/" + now.day();
	String time = String(now.hour()) + ":" + now.minute() + ":" + now.second();
	// Uses the time and date as arguments in the logdata function
	logdata(date, time);
}

void logdata(String date, String time) {
	Serial.println("Logging data...");
	// Gets data from temperature sensor and pH sensor
	float temperature = temp();
	float pHValue = pH();
	float uvValue = UV();
	float pressureValue = pressure();
	/* For debugging
	Serial.print("Temperature: ");
	Serial.println(temperature);
	Serial.print("pH Value read: ");
	Serial.println(pHValue);
	Serial.print("UV Index Reading: ");
	Serial.println(uvValue);
	Serial.print("Pressure: ");
	Serial.println(pressureValue);
	Serial.print("Time: ");
	Serial.println(time);
	*/
	
	// .csv format
	writeToSDCard(date + ";" + time + ";" + temperature + ";" + pHValue + ";" + uvValue + ";" + pressureValue + "\n");  // Can add Humidity, Soil Moisture, Light Intensity, Water Level
}

bool writeToSDCard(String data) {
	dataFile = SD.open(dataFileName, FILE_WRITE);
	Serial.println(data);
	// Check if the file opened successfully
	if (dataFile) {
		// Write data to the file
		dataFile.print(data);
		Serial.println("Data written to the file.");
		dataFile.flush();  // Flush the data to the SD card
		// Close the file
		dataFile.close();
		return true;
	} else {
		Serial.println("Error opening the file.");
		Serial.print("Filename: ");
		Serial.println(dataFileName);
		return false;
	}
}
// This prints the header to the SD card
void header() {
	int attempts = 0;
	while (!writeToSDCard("Date;Time;Temperature;pH Value;UV Index;Pressure (hPa)\n")) {
		attempts++;
		// If the function could not print to the SD card after 10 tries, then it gives up
		if (attempts > 10) {
			Serial.println("Could not print the header to the SD card");
			return;  // Exits the function
		}
		delay(500);  // Adds a delay of 0.5 sec before retrying
	}
	Serial.println("Header printed successfully");
}

float temp() {
	// Failed analog temperature sensor
	/*
	// Microvolts input from the termistor
	int resistance = analogRead(termistorPin);
	// Conversion to Celsius
	float temperature = (-0.1637931 * resistance) + 169.67;
	Serial.println(resistance);
	Serial.println(temperature);
	*/
	
	// Digital temp sensor
	tempSensors.requestTemperatures();
	float tempC = tempSensors.getTempCByIndex(0);
	
	return tempC;
}

float pH() {  
	/* Failed temperature sensor on pH sensor (TO pin)
	float temp = ph4502c.read_temp();
	Serial.println("pH temp: " + temp); 
	End test */
	
	float pH = ph4502c.read_ph_level();
	return pH;
}

float UV() {
	return uv.readUVI();
}

float pressure() {
	return baro.getPressure();
}
		\end{lstlisting}